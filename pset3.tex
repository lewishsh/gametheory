\documentclass[12pt]{article}
\usepackage[margin=1.25in]{geometry}
\usepackage{amsmath}
\usepackage{amssymb}
\usepackage{enumitem}
\usepackage{microtype}
\begin{document}
\begin{center}
  Lewis Ho\\
  Cooperative Game Theory\\
  Problem Set 3
\end{center}

\paragraph{Problem 1}
Notation: the strategies adopted by both players will be given by $(p,q)$,
where $p$ denotes the probability the row player picks the first option and
$q$ the probability the column player picks the first.
\begin{enumerate}[label=(\alph*)]
\item The zero sum game for this negotiation is $
  \begin{pmatrix}
    1 & 4\\
    0 & -3
  \end{pmatrix}.
  $
  Because $p = 1$ is a dominant strategy for player 1 (row), the nash equilibrium
  for this game is given by the threats $(1,1)$, with $D(1,1) = (3,2)$. $\sigma =
  6$, with $(1,0)$, so $R = (3.5, 2.5)$ with player 1 paying 1.5 to 2.
\item The zero sum game for this negotiation is $
  \begin{pmatrix}
    1 & -3\\
    0 & 4
  \end{pmatrix}
  $. Solving the equations $q - 3(1-q) = 4(1-q)$ and $p = -3p+4(1-p)$ gives
  the disagreement point $D(1/2,7/8) = (1.6875,1.1875)$. The TU solution involves
  playing $(0,0)$, and player 1 pays player 2 1.75 for $R = (3.25, 2.75)$.
\item The zero sum game is $
  \begin{pmatrix}
    3 & -6\\
    -1 & -1
  \end{pmatrix}
  $. 0 is a dominant strategy for the column player, giving us $D(0,0) = (5,6)$,
  which is also the profit maximizing set of strategies, thus no transfers need
  to be made.
\item The zero sum game is $
  \begin{pmatrix}
    -1 & 3\\
    -3 & 7
  \end{pmatrix}
  $. 1 is a dominant strategy for 2, thus we have $D(1,1) = (-1,0)$. The TU
  involves playing $(0,0)$, with 1 paying 4 for $R = (3,4)$.

\end{enumerate}

\paragraph{Problem 2}
The matrix for the zero sum game is:
\begin{displaymath}
  \begin{pmatrix}
    -2 & 2 & 0\\
    1 & -1 & -1\\
    0 & 1 & 0
  \end{pmatrix}.
\end{displaymath}
$(3,3)$ is a nash equilibrium for this game (denoting the row/column of the game),
and it is also the profit maximizing move. Thus $D = R = (3,3)$. The graph is
as follows:
\vspace{100pt}

\paragraph{Problem 3}
False: consider the counterexample with the game
\begin{displaymath}
  \begin{pmatrix}
    3,3 & 0,1\\
    0,2 & 0,1
  \end{pmatrix}.
\end{displaymath}
The zero sum game for this is
\begin{displaymath}
  \begin{pmatrix}
    0 & -1\\
    -2& -1
  \end{pmatrix},
\end{displaymath}
with a nash equilibrium at $D(0,0) = (0,1)$. With this disagreement point,
the TU value is actually $(2.5,3.5)$.
\end{document}