\documentclass[12pt]{article}
\usepackage[margin=1.25in]{geometry}
\usepackage{amsmath}
\usepackage{amssymb}
\usepackage{enumitem}
\usepackage{microtype}
\begin{document}
\begin{center}
  Lewis Ho\\
  Cooperative Game Theory\\
  Problem Set 2
\end{center}

\paragraph{Problem 1}
\begin{enumerate}[label=\alph*)]
\item The core is the set of points in $\mathbb{R}^3$ satisfying:
  \begin{align*}
    x_1 + x_2 + x_3 &= 1\\
    x_1 \leq \frac{1}{3},\ 
    x_2 &\leq \frac{1}{3},\ 
    x_3 \leq \frac{1}{2}
  \end{align*}
The shapley values are:
\begin{align*}
  x^*_1 &= \frac{1}{6}(0 + 0 + \frac{1}{2} + \frac{2}{3} + \frac{1}{3} + \frac{1}
          {3}) = 0.30\bar{5}\\
  x^*_2 &= \frac{1}{6}(0 + 0 + \frac{1}{2} + \frac{2}{3} + \frac{1}{3} + \frac{1}
          {3}) = 0.30\bar{5}\\
  x^*_3 &= \frac{1}{6}(0 + 0 + \frac{2}{3} + \frac{2}{3} + \frac{1}{2} + \frac{1}
          {2}) = 0.3\bar{8}
\end{align*}
\item The other values keep the following conditions the same:
  \begin{align*}
    x_1 + x_2 + x_3 &= 1\\
    x_1 \leq \frac{1}{3},\ 
    x_2 &\leq \frac{1}{3}
  \end{align*}
  $x$ gives us a upper bound on what we can offer 3. Clearly the most we can
  offer them, given our other upper bounds, is 1/3. Therefore if $x \geq 2/3$,
  the core will be nonempty.
\end{enumerate}

\paragraph{Problem 2}
Because this game is symmetric, our requirement for $y$ is that $y/3 \geq
(3/4)/2$. Thus $y \geq 9/8$ produces a nonempty core.

\paragraph{Problem 3}
Core $v = \{x\in \mathbb{R}^n_+\ :\ \sum_{i\in T(v)}x_i = 1\}$ is equivalent to
saying $x_i = 0$ for all $v\notin T(v)$, and that all combinations that satisfy
this are in the core. To show this condition is necessary, suppose it is false,
i.e we have an allocation where some $x_i \neq 0$ where $i\notin T(v)$. Because
$i$ is not in $T(v)$, there exists some coalition $S$ where $v(S) = 1$. This
coalition can block the allocation: we take $i$'s share, and the share of
anyone else not in $T(V)$, and distribute it amongst $S$. The share of $v(S)$
adds to one so it's feasible, and everyone is better than off in the new
allocation. I.e. $x_i = 0$ for all $i \notin T(v)$ is a necessary condition.

To show sufficiency, suppose we have an allocation where only those in $T(v)$
have nonzero payoffs. Because of this fact, any blocking coalition must
contain everyone with nonzero payoff, but in this case the coalition cannot
distribute the payoff in a way that is better for everyone in the coalition:
for someone to be better off, we must take away from someone with a nonzero
payoff in the original allocation, but all such people are also in the coalition,
so no such superior allocations exist for the coalition, and original allocation
is stable.

\paragraph{Problem 4}
I will assume this means that any coalition without the landlord produces 0
units of food.

For the landlord, note that we can account for all $S$ without the landlord by
counting all $S$ of different sizes, as all workers are the same. For each $s$,
there are 10 choose $s$ coalitions of that size, and $s$ ranges from 1 to 10
(we leave out zero because $f(0) = 0$).
Finally, note that $v(S\cup L) - v(S) = f(s)$, where $L$ is the landlord, as
the workers by themselves produce nothing. Thus the Shapley value for $L$ is:
\begin{displaymath}
  \sum_{s=1}^{10}\binom{10}{s}\frac{s!(11-s-1)!}{11!}f(s)
  = \sum_{s=1}^{10}\frac{10!}{s!(10-s)!}\frac{s!(10-s)!}{11!}f(s)
  = \sum_{s=1}^{10}\frac{f(s)}{11}.
\end{displaymath}

The other Shapley values can be calculated by efficiency and symmetry. If
$\phi_L(v)$ is the Shapley value for the landlord calculated above, then
$\phi_i(v)$ where $i \neq L$ is $(f(10)-\phi_1(v))/10$.

%Next, for the Shapley value for a worker, note that we need not count $S$ not
%including the landlord, because $f(S) = f(S \cup i) = 0$. Note further that all
%$S$ including the landlord can be characterized by its size $s$, and for each
%$s$ there are 9 choose $s-1$ of them (ways of choosing $s-1$ workers out of the
%9 who are not $i$). Finally, note that $v(S) = f(s-1)$ and $v(S\cup i) = f(s)$,
%again because each has $s-1$ and $s$ workers respectively. Thus the Shapley
%value is:
%\begin{align*}
%  &\sum_{s=1}^{10}\binom{9}{s-1}\frac{s!(11 - s - 1)}{11!}(f(s)-f(s-1))\\
%  = &\sum_{s=1}^{10}\frac{9!}{(s-1)!(10-s)!}\frac{s!(10-s)!}{11!}(f(s)-f(s-1))\\
%  = &\sum_{s=1}^{10}\frac{s}{110}(f(s)-f(s-1)),
%\end{align*}
%where we define $f(0) = 0$.

\paragraph{Problem 5}
In the first game, a coalition wins a majority if and only if it contains the
first player. Thus for $i=1$, $v(S\cup i)- v(S) = 1$, and $= 0$ for all others.
Thus the Shapley value $\phi_i(v) = 1$ for $i=1$, and $0$ otherwise.
For the second, let's consider $\phi_1(v)$ first. Of all the different ways of
ordering 1, 2, 3, 4 and 5, there are 24 orderings in which $v(S\cup i) - v(S) =
0$, namely those in which 1 comes last. It is equal to 1 in all other cases,
and thus the Shapley value is:
\begin{displaymath}
  \phi_1(v) = \frac{5!-24}{5!} = 0.8
\end{displaymath}
Then, by efficiency and symmetry, $\phi_i = 0.05$ for $i = 2,\ 3,\ 4,\ 5$.

\paragraph{Problem 6}
For permanent members, the coalitions $S$ for which $v(S\cup i) - v(S) = 1$
are those including all other permanent members and at least 4 non-permanent
members, say $k$ of them (this value is 0 for all other coalitions).
For any $k$, there are 4 choose 4 times 10 choose $k$
coalitions $S$ of that specification. Thus the Shapley value for a security
council member is:
\begin{displaymath}
  \sum_{k=4}^{10}\binom{10}{k}\frac{(4+k)!(15-(4+k)-1)!}{15!},
\end{displaymath}
which, with a little help from python, evaluates to 0.196. Then, using symmetry
and efficiency, we infer that the Shapley value for non-permanent members is
$(1-5\cdot0.195)/10 = 0.00186$.

\end{document}