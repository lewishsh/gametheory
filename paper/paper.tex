\documentclass[12pt]{article}
\usepackage[margin=1in]{geometry}
\usepackage{setspace}
\usepackage{amsmath}
\usepackage{amssymb}
\usepackage{booktabs}
%\doublespacing

\title{Dissolving a Partnership}
\author{Lewis Ho}

\begin{document}

\maketitle

\subsection*{Introduction}

John Harsanyi, in the 1959 paper ``A bargaining model for the cooperative n-person
game,'' proposed a procedure for dissolving a partnership in which the overall
value of the partnership is more than the sum of what they each contributed to it.
Given that each partner may have brought something of a different worth to the
table, and different groups within the partnership may also have made their own
collective contributions, the question of how the overall assets are to be
distributed to each is a complicated one.

Harsanyi's approach rests on the intuition that each partner, or subgroup,
should recoup what they contributed, and what remains should be distributed
evenly to all. His procedure, however, appears to be a fairly inexact
implementation of this idea: coalitions are chosen at random, their contributions
are distributed evenly and subtracted from the remaining assets to be distributed,
and this process continues until there is nothing left. 

There is, in fact, intention behind this apparently haphazard approach: 
the procedure works no matter how one decides to pick coalitions, and the outcome
always coincides exactly with the ideal put forward, known in game theory as
the Shapley value.
In this paper I describe and demonstrate the procedure and prove
that the outcome coincides with the Shapley value of the partnership.

\subsection*{The Procedure}

I formalize the notion of a partnership and its contributors as follows: let
$N = \{1,2,\ldots, n\}$ be partners, and let $v(S)$ with $S \subseteq N$ be
the worth or contribution of all members of $S$ collectively. This means, for
example, that $v(1,2)$ is the sum of all the contributions of $1$ and 2 as
individuals, and of their contribution as a pair. Thus $v(N)$ is the total
worth of the partnership being dissolved, and is the amount that must be divided
amongst the partners.
Note that it is
possible for the contributions of a collective to be negative, or less than
the sum of its parts, because the collective may have made a poor decision,
so both $v(A) + v(B) \leq v(A\cup B)$ and $v(A) + v(B) \geq v(A\cup C)$ are
possible. Finally, note that $v$ is a vector in $\mathbb{R}^{P(N)}$, where
$P(N)$ is the power set of $N$ (we exclude the null set from $P(N)$ for
convenience here), and that other ``games,'' i.e. other situations, can be
obtained for the same set of partners with other vectors $v'\in \mathbb{R}^{P(N)}$.

The procedure works as follows: any coalition $S\subset N$ is chosen, $v(S)$
is divided equally among the members of $S$, and $v(S)$ is subtracted from all
components of $v$ containing $S$ to yield some $v_2$.\footnote{As in $v_2(R) =
  v(R)-v(S)$ if $S \subseteq R$, and $v(R)$ otherwise.} This procedure repeats,
with $v_2, v_3\ldots v_k$ replacing $v$ and coalitions $S$ with $v_i(S)\neq 0$
selected until some $v_k = 0$ is reached.

\paragraph{Example:}
I dissolve the partnership represented by:
\begin{table}[h]
\centering
\begin{tabular}{|l|l|l|l|l|l|l|l|l|l|l|l|l|l|l|l|}
\hline
$S$  & 1  & 2  & 3  & 4  & 12 & 13 & 14  & 23  & 24 & 34 & 123 & 124 & 134 & 234 & 1234 \\ \hline
$v(S)$& 24 & 48 & 24 & 72 & 96 & 0  & 120 & 144 & 0  & 0  & 0   & 0   & 168 & 0   & 240  \\ \hline
\end{tabular}
\end{table}

I use the same notation as in the reading. In this first round I simply choose
coalitions from smallest to largest and in numerical order.
\begin{table}[h]
\centering
\scriptsize
\begin{tabular}{|l|l|l|l|l|l|l|l|l|l|l|l|l|l|l|l|l|l|l|l|}
\hline
$S$ & \multicolumn{15}{l|}{Game}                                                              & \multicolumn{4}{l|}{Allocations} \\ \hline
         & 1  & 2  & 3  & 4  & 12 & 13  & 14  & 23  & 24   & 34  & 123  & 124  & 134 & 234  & 1234 & 1      & 2      & 3      & 4     \\ \hline
         & 24 & 48 & 24 & 72 & 96 & 0   & 120 & 144 & 0    & 0   & 0    & 0    & 168 & 0    & 240  & 0      & 0      & 0      & 0     \\ \hline
1        & 0  & 48 & 24 & 72 & 72 & -24 & 96  & 144 & 0    & 0   & -24  & -24  & 144 & 0    & 216  & 24     & 0      & 0      & 0     \\ \hline
2        & 0  & 0  & 24 & 72 & 24 & -24 & 96  & 96  & -48  & 0   & -72  & -72  & 144 & -48  & 168  & 0      & 48     & 0      & 0     \\ \hline
3        & 0  & 0  & 0  & 72 & 24 & -48 & 96  & 72  & -48  & -24 & -96  & -72  & 120 & -72  & 144  & 0      & 0      & 24     & 0     \\ \hline
4        & 0  & 0  & 0  & 0  & 24 & -48 & 24  & 72  & -120 & -96 & -96  & -144 & 48  & -144 & 72   & 0      & 0      & 0      & 72    \\ \hline
12       & 0  & 0  & 0  & 0  & 0  & -48 & 24  & 72  & -120 & -96 & -120 & -168 & 48  & -144 & 48   & 12     & 12     & 0      & 0     \\ \hline
13       & 0  & 0  & 0  & 0  & 0  & 0   & 24  & 72  & -120 & -96 & -72  & -168 & 96  & -144 & 96   & -24    & 0      & -24    & 0     \\ \hline
14       & 0  & 0  & 0  & 0  & 0  & 0   & 0   & 72  & -120 & -96 & -72  & -192 & 72  & -144 & 72   & 12     & 0      & 0      & 12    \\ \hline
  23
    & 0  & 0  & 0  & 0  & 0  & 0   & 0   & 0   & -120 & -96 & -144 & -192 & 72  & -216 & 0    & 0      & 36     & 36     & 0     \\ \hline
24       & 0  & 0  & 0  & 0  & 0  & 0   & 0   & 0   & 0    & -96 & -144 & -72  & 72  & -96  & 120  & 0      & -60    & 0      & -60   \\ \hline
34       & 0  & 0  & 0  & 0  & 0  & 0   & 0   & 0   & 0    & 0   & -144 & -72  & 168 & 0    & 216  & 0      & 0      & -48    & -48   \\ \hline
123      & 0  & 0  & 0  & 0  & 0  & 0   & 0   & 0   & 0    & 0   & 0    & -72  & 168 & 0    & 360  & -48    & -48    & -48    & 0     \\ \hline
124      & 0  & 0  & 0  & 0  & 0  & 0   & 0   & 0   & 0    & 0   & 0    & 0    & 168 & 0    & 432  & -24    & -24    & 0      & -24   \\ \hline
134      & 0  & 0  & 0  & 0  & 0  & 0   & 0   & 0   & 0    & 0   & 0    & 0    & 0   & 0    & 264  & 56     & 0      & 56     & 56    \\ \hline
1234     & 0  & 0  & 0  & 0  & 0  & 0   & 0   & 0   & 0    & 0   & 0    & 0    & 0   & 0    & 0    & 66     & 66     & 66     & 66    \\ \hline
\multicolumn{16}{|l|}{Total}                                                                       & 74     & 30     & 62     & 74    \\ \hline
\end{tabular}
\end{table}

Thus the vector of allocations is $(74, 30, 62, 74)$.
And with a different order of coalitions chosen we get the same result:

\begin{table}[h]
  \scriptsize
\centering
\begin{tabular}{|l|l|l|l|l|l|l|l|l|l|l|l|l|l|l|l|l|l|l|l|}
\hline
$S$     & \multicolumn{15}{l|}{Game}                                                              & \multicolumn{4}{l|}{Allocations} \\ \hline
     & 1  & 2  & 3  & 4  & 12 & 13  & 14  & 23  & 24   & 34  & 123  & 124  & 134 & 234  & 1234 & 1      & 2      & 3      & 4     \\ \hline
     & 24 & 48 & 24 & 72 & 96 & 0   & 120 & 144 & 0    & 0   & 0    & 0    & 168 & 0    & 240  & 0      & 0      & 0      & 0     \\ \hline
3    & 24 & 48 & 0  & 72 & 96 & -24 & 120 & 120 & 0    & -24 & -24  & 0    & 144 & -24  & 216  & 0      & 0      & 24     & 0     \\ \hline
2    & 24 & 0  & 0  & 72 & 48 & -24 & 120 & 72  & -48  & -24 & -72  & -48  & 144 & -72  & 168  & 0      & 48     & 0      & 0     \\ \hline
4    & 24 & 0  & 0  & 0  & 48 & -24 & 48  & 72  & -120 & -96 & -72  & -120 & 72  & -144 & 96   & 0      & 0      & 0      & 72    \\ \hline
1    & 0  & 0  & 0  & 0  & 24 & -48 & 24  & 72  & -120 & -96 & -96  & -144 & 48  & -144 & 72   & 24     & 0      & 0      & 0     \\ \hline
23   & 0  & 0  & 0  & 0  & 24 & -48 & 24  & 0   & -120 & -96 & -168 & -144 & 48  & -216 & 0    & 0      & 36     & 36     & 0     \\ \hline
13   & 0  & 0  & 0  & 0  & 24 & 0   & 24  & 0   & -120 & -96 & -120 & -24  & 96  & -216 & 48   & -24    & 0      & -24    & 0     \\ \hline
24   & 0  & 0  & 0  & 0  & 24 & 0   & 24  & 0   & 0    & -96 & -120 & -24  & 96  & -96  & 168  & 0      & -60    & 0      & -60   \\ \hline
14   & 0  & 0  & 0  & 0  & 24 & 0   & 0   & 0   & 0    & -96 & -144 & -48  & 72  & -96  & 144  & 12     & 0      & 0      & 12    \\ \hline
12   & 0  & 0  & 0  & 0  & 0  & 0   & 0   & 0   & 0    & -96 & -144 & -72  & 72  & -96  & 120  & 12     & 12     & 0      & 0     \\ \hline
34   & 0  & 0  & 0  & 0  & 0  & 0   & 0   & 0   & 0    & 0   & -144 & -72  & 168 & 0    & 216  & 0      & 0      & -48    & -48   \\ \hline
124  & 0  & 0  & 0  & 0  & 0  & 0   & 0   & 0   & 0    & 0   & -144 & 0    & 168 & 0    & 288  & -24    & -24    & 0      & -24   \\ \hline
134  & 0  & 0  & 0  & 0  & 0  & 0   & 0   & 0   & 0    & 0   & -144 & 0    & 0   & 0    & 120  & 56     & 0      & 56     & 56    \\ \hline
123  & 0  & 0  & 0  & 0  & 0  & 0   & 0   & 0   & 0    & 0   & 0    & 0    & 0   & 0    & 264  & -48    & -48    & -48    & 0     \\ \hline
1234 & 0  & 0  & 0  & 0  & 0  & 0   & 0   & 0   & 0    & 0   & 0    & 0    & 0   & 0    & 0    & 66     & 66     & 66     & 66    \\ \hline
\multicolumn{16}{|l|}{Total}                                                                   & 74     & 30     & 62     & 74    \\ \hline
\end{tabular}
\end{table}

I proceed to prove two propositions about the procedure.

\subsection*{Termination}

I first show that the procedure terminates regardless of which coalitions are
chosen at each step. The proof proceeds by contradiction. Suppose the statement
is false,
then there exists an infinite sequence of coalitions $\{S_n\}_{n=1}^\infty$ such
that $v_{n}(S_n)\neq 0$, with (for convenience we let $v_1 = v$, the original
game)---as the procedure terminates when no coalition with nonzero value exists.

As $|P(N)|$ is finite, there must be coalitions appearing infinitely many times
in $\{S_n\}$, and there must be a smallest (though not necessarily strictly
smallest) coalition appearing infinitely many times. Note then two facts:
\begin{enumerate}
\item During some round $i$ of a procedure, for all $S \subseteq N$ with $|S|
  \leq |S_i|$ and $S \neq S_i$, $v_{i+1}(S) = v_i(S)$,
  where $S_i$ is the chosen coalition for that
  round. This is because given those conditions $S_i \not\subseteq S$, thus
  $v_{i+1}(S) = v_i(S)$ by the definition of $v_{i+1}$.
\item If $S_i$ is the chosen coalition in round $i$, $v_{i+1}(S_i) = 0$.
\end{enumerate}
Thus if $S$ is the smallest coalition that repeats infinitely, then there is some
point in $\{S_n\}$ in which all the coalitions subsequently chosen are either the
same size or greater than $S$. Thus the next time $S$ is chosen, say in turn
$i$, $v_{i+1}(S) = 0$, and because each subsequent $S_k$ is of equal size or
larger than $S$, $v_k(S)$ remains as 0 for all $k > i$, and thus cannot be chosen
again, contradicting the assumption that $S$ repeats infinitely. Thus our
assumption cannot be true, i.e. our procedure must always terminate.

\subsection*{Uniqueness and the Shapley Value}


Another useful concept for considering the distribution of value across a
collective is the Shapley value, a quantity that allocates amongst members of a
group each individual/coalition's contribution and then divides the remainder
evenly. This is the ideal distribution that realizes Harsanyi's intuition.
In fact, I show in this section that the outcome of the
procedure is the same regardless of which coalition is chosen at any stage,
and that the outcome is always the Shapley value.

\paragraph{Example:}

To first demonstrate this correspondence, I shall calculate the Shapley value
for each partner in the example given above. Instead of working with the formula
I simply generate all the permutations of $[1,2,3,4]$ and calculate $v(S) -
v(S\cup i)$, add it to the $i$th component of my vector, then divide by 4! = 24. 
I wrote a python script to do this:

\begin{verbatim}
import numpy as np
import itertools

v = {():0, (1,):24, (2,):48, (3,):24, (4,):72, (1,2):96, (1,3):0, (1,4):120,
     (2,3):144, (2,4):0, (3,4):0, (1,2,3):0, (1,2,4):0, (1,3,4):168,
     (2,3,4):0, (1,2,3,4):240}
permutations = list(itertools.permutations([1,2,3,4]))
shapley = [0,0,0,0]

for permutation in permutations:
    S = []
    for i in permutation:
        S_old = S[:]
        S += [i]
        S.sort()
        shapley[i-1] += v[tuple(S)] - v[tuple(S_old)]

shapley = np.array(shapley)/24
print(shapley)
\end{verbatim}
The result is \texttt{[74. 30. 62. 74.]}, which is exactly what I got from both
uses of the procedure.

\paragraph{Proof:}
I now show that the procedure always gives each partner their Shapley value.
Let $u_S\in \mathbb{R}^{P(N)}$ be a vector/game with components $(u_S)_T = 1$ if
$S \subseteq T$ and 0 otherwise. Note that at each step $i$, we generate our
new game $v_{i+1}$ by subtracting $u_{S_i}\cdot v_i(S_i)$ from $v_{i}$.
With this we can express the games $\{v_i\}$
resulting from each stage of the procedure recursively:
\begin{align*}
  v_2 &= v_1 - u_{S_1}\cdot v_1(S_1)\\
  v_3 &= v_2 - u_{S_2}\cdot v_2(S_2) = v_1 - u_{S_1}\cdot v_1(S_1) - u_{S_2}\cdot
        v_2(S_2)\\
      & \vdots\\
  v_k &= v_{k-1} - u_{S_{k-1}}\cdot v_{k-1}(S_{k-1}) = v_1 - \sum_{i=1}^{k-1}
        u_{S_i}\cdot v_i(S_i).
\end{align*}
Because the procedure terminates after some turn $k-1$, we know the resultant
game $v_k = 0$. Thus, rearranging our last equation, we get
\begin{displaymath}
  \sum_{i=1}^{k-1}u_{S_i}\cdot v_i(S_i) = v_1 = v.
\end{displaymath}
By the additivity of the Shapley value, we have:
\begin{align}
  \phi(v) = \phi\left(\sum_{i=1}^{k-1}u_{S_i}\cdot v_i(S_i)\right) =
  \sum_{i=1}^{k-1}\phi(u_{S_i})\cdot v_i(S_i).
\end{align}

To complete the proof, I write the allocations from each round in a similar
manner. In each turn $i$, if $S_i$ is the chosen coalition, each player $j$
gets $v_i(S_i)/|S_i|$ if they are in $S_i$ and nothing otherwise. Note then that
in the game $u_{S_i}$, all $j\notin S_i$ are dummies and the rest are symmetric,
and thus $\phi(u_{S_i})_j = 1/|S_i|$ for $j\in S_i$ and zero otherwise. Putting
the two together, the allocation for each player $j$ in round $i$ can be written
as $\phi(u_{S_i})_j\cdot v_i(S_i)$. Over the course of the $k-1$ turns of the
procedure, each player $j$ thus receives
\begin{displaymath}
  \sum_{i=1}^{k-1}\phi(u_{S_i})_j\cdot v_i(S_i) = \left(\sum_{i=1}^{k-1}\phi(u_{S_i})
    \cdot v_i(S_i)\right)_j = \phi(v)_j
\end{displaymath}
by equation (1). In other words each player receives his Shapley value, thus
completing the proof.

\subsection*{Conclusion}

The procedure for dissolving a partnership examined above appears to be a
haphazard implementation of the idea that each individual/coalition should
receive what they put in, and the surplus allocated evenly to those who have a
claim to it. However, from the examination above it is clear that this seemingly
random process is actually an exact implementation of that intuition in that
it always terminates no matter how it is used, and it always delivers the desired
outcome, namely the Shapley value of the partnership.

\end{document}